\section{Introduction}
The rapid proliferation of social media platforms, particularly Twitter (now known as X), has revolutionized the way information is disseminated and consumed globally. Social media serves as a real-time pulse of public opinion, breaking news, and emerging trends, making it a vital source of information for various fields, including journalism, public safety, crisis management, and disaster response. The vast and dynamic nature of social media content, however, presents significant challenges in identifying and summarizing relevant events effectively.

Automated journalism, which aims to generate news content from structured data with minimal human interference, has emerged as a promising solution to these challenges. Central to this concept is the ability to monitor social media, detect relevant events in real-time, and generate accurate, concise summaries that can be used directly in news reporting. This paper presents our work on developing the Social Media Monitoring module that integrates with the Automated Journalist App, designed to automatically detect and cluster events from social media data.

Our work builds on the EDT\textsubscript{BERT} framework by \citet{edtbert}, incorporating both structural and contextual embeddings to preprocess tweets and represent them as nodes in a graph, with edges denoting similarities. By applying the Markov Clustering (MCL) algorithm \cite{mcl}, we identify clusters of related tweets, which represent potential events. To further enhance the system's utility, we use a large language model (LLM) to generate summaries of these clusters, making the detected events more accessible and actionable for automated news generation.

Our approach is evaluated using the Event2012 dataset, where it achieves a precision of 85\%, a recall of 70\%, and an F1-score of 77\%, demonstrating the system's robustness and effectiveness in real-time event detection. This work significantly contributes to the field of automated journalism by providing a reliable method for real-time social media monitoring and event summarization.

The Social Media Monitoring project was a collaborative effort by the authors of this paper. Aysenur Kocak was responsible for implementing Contextual Embeddings (Section \ref{sec:contextualembeddings}), Graph Clustering (Section \ref{sec:graphclustering}), and Event Summarization (Section \ref{sec:eventsummarization}). She also conducted research on BERT-Based Event Detection (Section \ref{sec:bertbased}) and led the compilation of the poster and the report. Youssef Lotfy was responsible for implementing Structural Embeddings (Section \ref{sec:structuralembeddings}), Dashboard (Section \ref{sec:dashboard}) and code documentation. He also conducted research on Graph-Based Event Detection (Section \ref{sec:graphbased}). Leonhard Stengel ensured code quality and was responsible for the implementations of Preprocessing (Section \ref{sec:preprocessing}), Graph Construction (Section \ref{sec:graphconstruction}), and Integration to the Automated Journalist App (Section \ref{sec:integration}). He also conducted research on Clustering-Based Event Detection methods (Section \ref{sec:clusteringbased}).