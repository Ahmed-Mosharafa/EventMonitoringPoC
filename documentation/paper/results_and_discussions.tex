\section{Results \& Discussion}
The results of our study demonstrate the effectiveness of our Social Media Monitoring module, particularly in detecting and clustering relevant events from Twitter data in real-time. We evaluated our system on the Event2012 dataset. The system achieved a precision of 85\%, a recall of 70\%, and an F1-score of 77\%, which are comparable to or exceed the performance of previous state-of-the-art methods.
\subsection{Precision, Recall, and F1-Score}
The high precision (85\%) indicates that the system is highly accurate in identifying true events from the tweet clusters, with minimal false positives. This precision is crucial in applications like automated journalism, where the accuracy of detected events directly impacts the quality of the generated news content. The recall of 70\%, while slightly lower, suggests that the system is reasonably effective at capturing a wide range of events, though some relevant events might be missed, which is an area for potential improvement. The F1-score of 77\% reflects a balanced performance, combining both precision and recall into a single metric that underscores the robustness of our approach.
\subsection{Event Summarization and Utility}
The event summarization module, powered by Google’s FLAN-T5 model, effectively condensed the tweet clusters into concise and informative summaries. This is particularly valuable for automated journalism, where the speed and clarity of reporting are paramount. The summaries generated were coherent and contextually relevant, often capturing the essence of the events with minimal extraneous information. This level of summarization ensures that the outputs are not only accurate but also actionable, providing journalists and end-users with clear insights into emerging trends and breaking news.

\subsection{Dashboard and User Interaction}
The integration of the Event Detection Module with the Automated Journalist App, along with the implementation of a user-friendly dashboard, significantly enhances the utility of the system. The dashboard's ability to visualize events and their relevance in real-time offers users an intuitive way to monitor and assess the significance of detected events. The iterative algorithm for overlap resolution in visualization ensures that users can easily distinguish between different event clusters, improving the overall user experience.

\subsection{Challenges and Limitations}
Despite the system’s strengths, there are several challenges and limitations that warrant discussion. One of the main limitations is the reliance on a fixed threshold for similarity measures, which may not be universally optimal across different types of events or datasets. Future work could explore adaptive thresholding mechanisms to enhance flexibility and accuracy. Additionally, the recall rate suggests that some relevant events may be overlooked, possibly due to the granularity of clustering or the inherent noise in social media data. Addressing these issues may involve refining the preprocessing steps or exploring alternative clustering algorithms.

Another challenge was the dependency on tweet data for evaluation, as Twitter’s policy restrictions posed limitations on accessing and sharing comprehensive datasets. This constraint necessitated the use of unofficial datasets, which, while effective, may not fully capture the diversity and dynamics of real-world social media environments.

