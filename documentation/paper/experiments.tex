\section{Experiments}
Most event detection solutions have historically been developed and tested using Twitter data due to its real-time nature and extensive user base. Following this, we also evaluated our approach using tweets. However, there was a significant problem: Twitter’s policies do not allow sharing of tweet datasets with the tweet texts, only tweet IDs. Furthermore, accessing tweets through the Twitter API proved financially unfeasible. Fortunately, we discovered an unofficial dataset containing over 70,000 tweets, each labeled with the creation date, time, and a event the tweet is about. Before evaluation, we preprocessed the data by removing duplicates and converting the dataset into a CSV format. This enabled us to carry out an evaluation of the EDT\textsubscript{BERT} implementation on this dataset. To evaluate the clustering results, we used precision, recall, and F1 scores, which are essential metrics for determining the accuracy of classification systems. For each pair of tweets, clustering results were compared: if both tweets were in the same cluster and matched the ground truth, they were counted as True Positive; if in different clusters and correctly so in the ground truth, as True Negative. Conversely, mismatches were counted as False Positives or Negatives. Using these criteria, we calculated the precision, recall, and F1 score, providing a assessment of our model’s performance.